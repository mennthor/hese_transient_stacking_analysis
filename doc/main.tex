\documentclass[
  bibliography=totoc,     % Bibliography as unnumbered chapter in toc
  numbers=noenddot,       % No dot after figure/table number
  captions=tableheading,  % Correct spacing for table headings
  % titlepage=firstiscover, % Symmetrical margins on titlepage
  parskip=half,           % Hals skip instead of intend in new paragraph
  headings=normal,
  a4paper,
]{scrartcl}

% Warning, if another latex run is needed
\usepackage[aux]{rerunfilecheck}
\setcounter{tocdepth}{2}

%------------------------------------------------------------------------------
%-- Language and Type
%------------------------------------------------------------------------------
\usepackage{fontspec}
\defaultfontfeatures{Ligatures=TeX}  % -- becomes en-dash etc.

% Language: english | german
\usepackage{polyglossia}
\setdefaultlanguage{english}

% For german | english abstract and german | english titles in the toc
\setotherlanguages{english}

% Intelligent quotation marks, language and nesting sensitive
\usepackage[autostyle]{csquotes}

% Microtypographical features, makes the text look nicer on the small scale
\usepackage{microtype}

%------------------------------------------------------------------------------
%-- Math
%------------------------------------------------------------------------------
\usepackage{amsmath}
\usepackage{amssymb}
\usepackage{mathtools}

% Enable Unicode-Math and follow the ISO-Standards for typesetting math
\usepackage[
  math-style=ISO,
  bold-style=ISO,
  sans-style=italic,
  nabla=upright,
  partial=upright,
]{unicode-math}
\setmathfont{Latin Modern Math}

\usepackage{xfrac}   % Small fracs for the text with \sfrac{}{}
\usepackage{braket}  % Good for expectation values

%------------------------------------------------------------------------------
%-- Numbers and Units
%------------------------------------------------------------------------------
\usepackage[
  locale=US,
  separate-uncertainty=true,
  per-mode=symbol-or-fraction,
  exponent-product=\cdot,
]{siunitx}
\sisetup{math-micro=\text{µ},text-micro=µ}

%------------------------------------------------------------------------------
%--Tables
%------------------------------------------------------------------------------
\usepackage{booktabs}  % Use \toprule, \midrule, \bottomrule

%------------------------------------------------------------------------------
%-- Graphics
%------------------------------------------------------------------------------
\usepackage{graphicx}
\usepackage{grffile}

%------------------------------------------------------------------------------
%-- Colors
%------------------------------------------------------------------------------
\usepackage{xcolor}
\xdefinecolor{darkgrey}{HTML}{353132}
% Nord colors: https://github.com/arcticicestudio/nord
% Polar Night (dark greys)
\xdefinecolor{nordgrey1}{HTML}{2E3440}
\xdefinecolor{nordgrey2}{HTML}{3B4252}
\xdefinecolor{nordgrey3}{HTML}{434C5E}
\xdefinecolor{nordgrey4}{HTML}{4C566A}
% Snow Storm (muted whites)
\xdefinecolor{nordwhite1}{HTML}{D8DEE9}
\xdefinecolor{nordwhite2}{HTML}{E5E9F0}
\xdefinecolor{nordwhite3}{HTML}{ECEFF4}
% Frost (cool blues/greens)
\xdefinecolor{nordbluegreen}{HTML}{8FBCBB}
\xdefinecolor{nordcyan}{HTML}{88C0D0}
\xdefinecolor{nordpaleblue}{HTML}{81A1C1}
\xdefinecolor{norddeepblue}{HTML}{5E81AC}
% Aurora (muted accent colors)
\xdefinecolor{nordred}{HTML}{BF616A}
\xdefinecolor{nordorange}{HTML}{D08770}
\xdefinecolor{nordyellow}{HTML}{EBCB8B}
\xdefinecolor{nordgreen}{HTML}{A3BE8C}
\xdefinecolor{nordviolet}{HTML}{B48EAD}

%------------------------------------------------------------------------------
%-- Floats
%------------------------------------------------------------------------------
% Allow figures to be placed in the running text by default:
\usepackage{scrhack}
\usepackage{float}
\floatplacement{figure}{htbp}
\floatplacement{table}{htbp}

\usepackage[section, below]{placeins}  % Keep figures and tables in the section

\usepackage{caption}
\usepackage{subcaption}

%------------------------------------------------------------------------------
%-- Customize list environments
%------------------------------------------------------------------------------
\usepackage{enumitem}

%------------------------------------------------------------------------------
%-- Bibliography
%------------------------------------------------------------------------------
\usepackage[
  backend=biber,    % use modern biber backend
  autolang=hyphen,  % load hyphenation rules for if language of bibentry is not
                    % german, has to be loaded with \setotherlanguages
                    % in the references.bib use langid={en} for english sources
]{biblatex}

%------------------------------------------------------------------------------
%-- Code environment and verbatim
%------------------------------------------------------------------------------
\usepackage{listings}
\lstset{basicstyle=\ttfamily}
\newcommand{\code}[1]{\lstinline|#1|}

%------------------------------------------------------------------------------
%-- Misc
%------------------------------------------------------------------------------
\usepackage[pdfusetitle,unicode,]{hyperref}
\hypersetup{
  unicode=true,
  colorlinks,
  linkcolor=black,
  citecolor=black,
  filecolor=black,
  urlcolor=norddeepblue,
}
\usepackage{bookmark}
\usepackage[shortcuts]{extdash}
\usepackage{scrlayer-scrpage}  % For custom KOMA layout modifications

%------------------------------------------------------------------------------
%-- Custom math operators
%------------------------------------------------------------------------------

\newcommand{\mperiod}{\quad\text{.}}
\newcommand{\mcomma}{\quad\text{,}}
\newcommand{\mintertext}[1]{\quad\text{#1}\quad}

\renewcommand{\d}[1]{\mathrm{d}#1}
\newcommand{\del}[1]{\partial #1}
\newcommand{\dd}[2]{\frac{\d{#1}}{\d{#2}}}
\newcommand{\deldel}[2]{\frac{\partial #1}{\partial #2}}

\DeclareMathOperator{\rect}{rect}
\DeclareMathOperator{\diag}{diag}
\DeclareMathOperator{\trace}{trace}


% \addbibresource{references.bib}


\begin{document}

\begin{titlepage}
  \centering
  {\scshape\Large Analysis Note\par}
  \vspace{1.5cm}
  {\huge\bfseries Time Dependent Point Source Stacking Search Associated with HESE Track Events\par}
  \vspace{2cm}
  {\Large Thorben Menne\par}

  \vfill

  % Bottom of the page
  \textsc{Transients Working Group}\par
  \url{https://wiki.icecube.wisc.edu/index.php/Transients_Working_Group}\par
  \vspace{0.5cm}
  \textsc{Reviewers}\par
  \begin{tabular}{rl}
    \centering
    Collaboration: & Sandro Kopper (\href{mailto:sjkopper@ua.edu}{sjkopper@ua.edu}) \\
    Working Group: & Kevin Meagher (\href{mailto:kmeagher@ulb.ac.be}{kmeagher@ulb.ac.be}) \\
  \end{tabular}\par
  \vspace{0.5cm}
  \textsc{Contact}\par
  \begin{tabular}{rl}
    \centering
    Mail: & \href{mailto:thorben.menne@tu-dortmund.de}{thorben.menne@tu-dortmund.de} \\
    Slack: & @mennthor
  \end{tabular}\par
  \vspace{2cm}
  {\Large Last edit: \today\par}
\end{titlepage}


\section{Overview}
This analysis is searching for a stacked lower energy neutrino contribution at the HESE track event positions, which are treated as transient sources.
We test for 21 generic box time windows increasing in logarithmic time from 2 seconds to 5 days symmetrically around all sources.

For the physics motivation: IceCube couldn't find a significant point source so far despite many different analysis and efforts.
Prominent examples relevant for this is search are the \href{time integrated 7 year point source search} and the \href{HESE 6 year point source search}, both with no significant results.
But the HESE events on their own show a clear astrophysical signal and therefore should originate from some sources.
This analysis test if there is any clustering of lower energy neutrino events around the HESE locations, aiming to constrain various HESE emission models.
\textcolor{nordred}{Note: The model selection is currently reviewed}.

The analysis method is a time dependent unbinned likelihood approach similar to the so called \enquote{GRB Likelihood}.
Key features used here are:
\begin{itemize}
  \item Background is modeled using scrambled data
  \item Using time dependent spatial background PDFs
  \item Energy PDF is estimated from data and MC with fixed index $E^{-2}$ power law
\end{itemize}

For source and test data we use:
\begin{itemize}
  \item Source dataset: 6 years HESE track events, IC79 - IC86, 2015.
  \item Test dataset: 5 years PS tracks (IC79 - IC86, 2014) + 1 year GFU (IC86, 2015)
\end{itemize}

The wiki can be found at \url{https://wiki.icecube.wisc.edu/index.php/Transient_HESE_Stacking} and contains this analysis note.


\section{Software}
The analysis scripts are in a git repository and can be found at \path{/home/tmenne/analysis/hese_transient_stacking/analysis}.
Code is enumerated in the order of script execution, if someone wants to redo all analysis steps.
The branch for this analysis is \code{original_hese} which tests the 22 original HESE track events.

The main analysis software used is made from scratch in python with a small C++ back-end for time consuming work with inspiration from \href{http://code.icecube.wisc.edu/projects/icecube/browser/IceCube/sandbox/skylab}{skylab} and \href{http://code.icecube.wisc.edu/projects/icecube/browser/IceCube/sandbox/richman/grbllh}{grbllh}.
The software repository can be found at \path{/home/tmenne/software/tdepps}.
The branch used for the analysis is \code{new_structure}.

\section{Datasets}
This analysis used HESE track events as sources and point source and GFU samples as a test dataset.

\section{Analysis Method}

\section{Analysis Performance}

\section{Analysis Results}


\appendix
\part*{\appendixname}
\section{Presentations}
\begin{description}
  \item[Transients Call] May 22nd, 2017: \href{https://drive.google.com/file/d/0B_Gkg-MCR-1za1RMbjlzTFE0YVU/view}{First Presentation in Transients group}
  \item[Transients Call] June 12th, 2017: \href{https://drive.google.com/file/d/0B_Gkg-MCR-1zTFI3Umg3XzZrSE0/view}{Progress on Software}
  \item[Transients Call] September 18th: \href{https://drive.google.com/file/d/0B_Gkg-MCR-1zR28tTmhBT3VYTGs/view}{Updates on tests on John Felde's NRT analysis}
  \item[Fall Meeting Berlin] October 10th, 2017: \href{https://events.icecube.wisc.edu/getFile.py/access?contribId=37&sessionId=32&resId=0&materialId=slides&confId=90}{First studies on HESE events with PS sample}
  \item[Transients Call] October 30th, 2017: \href{https://drive.google.com/file/d/0B_Gkg-MCR-1zOFdkajczT3JWNUU/view}{Performance and BG trials for all time windows}
  \item[Transients Call] April 30th, 2018: \href{https://drive.google.com/file/d/12vOMOpt1nMrmnBdM_4wV5sMdg0FUJLqF/view}{Update to 6 years of HESE sources}
\end{description}

\end{document}